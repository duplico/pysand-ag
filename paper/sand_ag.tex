\documentclass{article}
\usepackage{listings}
\usepackage{microtype}
\author{George Louthan}
\title{Passive Network Model Generation For Attack Graphs With Signature 
	Based Automatic Network Protocol Detection}
\begin{document}
\maketitle
\abstract{
Attack graphs provide insights into the security of an entire network,
especially in highly heterogenous settings with a wide variety of systems and
attack surfaces whose interactions are difficult to intuit. However, their
use requires an appropriate formal specification of the network. The
creation of such a model by hand is difficult, error-prone, and time consuming.
This paper presents a tool that utilizes deep
packet inspection to generate network model specifications appropriate for use
in the style of attack graphs employed at the University of Tulsa. 
The generated model is far from complete, but it does 
provide a useful starting point from which modeling can proceed.
}
\section{Introduction}
Attack graphs are useful tools for analyzing network security, specifically
the interactions of network objects and vulnerabilities.
A graph theoretical modeling framework in which nodes are states and edges
are transitions caused by an adversary, the attack graph permits analysis 
of compound exposures across a network by
chaining attacks across an entire changing system, rather than by
analyzing each component or vulnerability in isolation. 

However, in order to produce an attack graph, an appropriate model of the
system or network must be provided. By hand, the process of creating such a 
specification can be error-prone
and time consuming. To alleviate this problem, this paper presents an
automated passive tool that utilizes deep packet inspection to build a network 
model of hosts and topologies reflecting the connections and protocols whose 
traffic has been identified, regardless of TCP port number used. Though the
resultant model is neither perfect nor comprehensive, it provides an excellent
starting point for a modeler.

Although the specification of exploit patterns is also necessary for the
successful employment of attack graphs, their procurement and generation is
not within the scope of this paper.

The remainder of this paper is organized as follows. Section 2 provides some
overview level background into the version of the attack graph formalism used
for this work, the network model required by that formalism, and a basic
lexicon of terms used for property and topology names, as well as an
introduction to the deep packet inspection tool used. Section 3 describes the
specifics of the network model generation methodology, and Section 4 draws
conclusions and suggests future work.
\section{Background}
\subsection{Attack Graphs}
Given a set of so-called
\emph{exploit patterns} and a network model that specifies the initial state
of the network, attack graphs use graph theory to explore the security properties
of a network by closing its state space under its realizable 
exploits~\cite{campbell2002modeling}~\cite{phillips1998graph}~\cite{sheyner2002automated}.

Due to space constraints, a thorough grounding in the type of attack graphs
employed is impossible; however, the reader is referred to a 2005 literature
review by Lippmann and Ingols~\cite{lippmann2005annotated} for a thorough
survey of the field. Although an attack graph requires both a network model
and a set of exploit patterns, this section describes only the network model
component, as it is the only part relevant to this paper.
\subsubsection{Network Model}
A network model specification consists of a list of \emph{assets}, which are the
security principals at play in the network (possibly including hosts, users, 
data, the attacker, and anything else that needs to be treated as a first class
object in the model) and an initial \emph{fact
base}, which consists of two types of facts: \emph{qualities}, which are name-value pairs
representing asset properties, and \emph{topologies}, which are named
one-way relationships between two assets. This section provides a prose
description of the network model specification language, which may be difficult
to understand without an example to reference; the reader is encouraged to
follow along with Fig.~\ref{ag_illustrative}, a meaningless but valid and 
illustrative example of the specification language.

In our specification language, a network model is begun with the phrase 
\texttt{network model} followed by the
\texttt{=} symbol. The first component of the network model is the list of
assets, which is begun with the word \texttt{assets} followed by a colon. 
Asset names are
semicolon separated. Following the asset list is the initial fact base. It
is begun with the word \texttt{facts} followed by a colon.

Quality facts begin with the word \texttt{quality}, followed by a colon,
followed by the name of the asset whose quality is being defined, then a comma,
then the name of the quality (an atom), then an assignment operator, then the 
value of the quality.

Topology facts begin with the word \texttt{topology}, followed by a colon,
followed by the name of the source asset, followed by either the one-way
topology symbol \texttt{->} or the two-way topology symbol \texttt{<->} (though
the two-way topology is actually processed into two one-way topology facts),
followed by the name of the destination asset, followed by a comma, followed
by the name of the topology.

Each fact is terminated with a semicolon, and the network model is concluded 
with a period.
\begin{figure}
\begin{lstlisting}
network model = 
    assets :
        asset_1;
        asset_2;
        asset_3;

    facts :
        quality:asset_1,quality_1=value_1;
        quality:asset_2,quality_1=value_2;        
        quality:asset_3,quality_1=value_2;
        topology:asset_1->asset_2,topology_1;
        topology:asset_2<->asset_3,topology_2;
.
\end{lstlisting}
\caption{Example attack graph network model specification}
\label{ag_illustrative}
\end{figure}
\subsubsection{Term Lexicon}
Because the names used in qualities and topologies are so freeform, a consistent
dictionary of terms is required to facilitate modeling. This section
presents three different types of terms: \emph{access topologies}, 
\emph{connectivity topologies}, and \emph{status qualities}.

Access topologies are based upon the types of access rights used in the
United States' National Vulnerability Database (NVD): user (referring to 
user level access to a system), root (referring to administrative level access 
to a system), and other (referring to some kind of application-specific but
nevertheless privileged type of system access)~\cite{nvdhome}. These
access types are represented with a topology whose source is the asset
who \emph{has} access, whose destination is the asset to whom the source has
access, and whose name is \texttt{access\_\emph{type}}. In the case where type
is \texttt{other}, it is followed by another underscore and the name of the
application whose access rights are in question.

Connectivity topologies are also due to NVD specification~\cite{nvdhome}. They
may be \texttt{connected\_local}, meaning a direct, local, physical connection;
\texttt{connected\_adjacent}, meaning that they are on the same network segment
or collision domain; or \texttt{connected\_network\_\textit{protocol}}, meaning
that they are connected across the Internet using the protocol whose lower case
abbreviation (e.g. \texttt{ssh} for Secure Shell or 
\texttt{http} for web) replaces \texttt{\textit{protocol}}.

Finally, the status quality refers to whether the host is available; 
availability is intended to be used as a precondition (possibly using a 
preprocessor) for every exploit, and unavailability is intended as a 
postcondition of denial of service attacks. The status quality has the name
\texttt{status}, and its value may be the token \texttt{up} or \texttt{down}.
\subsection{SAND}
Signature based automatic network protocol detection (SAND) is the name of
a tool and process due to Louthan, McMillan, Johnson, and Hale that uses
deep packet inspection (DPI) to passively identify the protocol employed by
network traffic regardless of port number~\cite{louthan2009toward}.

Implemented as a library called pysand, SAND provides the ability to sniff
from the wire or from stored files, and report to the driver program when
a new network stream is detected, when a stream's protocol is identified,
and when a stream is closed.
\section{Methodology}
The automated network model generation tool provides the following features.
Upon the detection of a new stream, it adds the server and client to the
network model and gives them the \texttt{up} status quality. When a stream
is identified, it gives the relevant assets a topology with the 
appropriate \texttt{connected\_network\_\textit{protocol}} name. It also
provides optional automated DNS lookup for asset names, manual translation from
IP addresses to asset names using a user-provided \emph{mapfile}, and also
provides sane defaults for the automated mapping of IP addresses to
valid asset names obeying the standard lexicon provided previously.

The procedure for generating a network model specification begins when the first
stream is detected. These streams' components are added to an intermediate
internal asset list as IP addresses only, and an intermediate quality fact
representation with the quality name \texttt{status} and value \texttt{up} is
also added to a separate data structure, which holds all
facts about the system and ensures the uniqueness of its
members.

Once a stream is identified, a topology fact is inserted into the
intermediate fact base; the client is the source, and the server is the
destination. The topology is named \texttt{connected\_network\_\textit{protocol}},
where \emph{\texttt{protocol}} is determined by the protocol abbreviation used
in the SAND identifier (by convention, the standard abbreviation).

Upon completion of execution, the network model is built by applying minimal
postprocessing to the intermediate asset and fact list. First, the mapfile (if
any) is consulted to map IP addresses to friendlier asset names. Then, if
the mapfile specifies that DNS is to be used, each IP address is queried
against the local DNS server to attempt to obtain a hostname. If neither of
these yield an alternative to the IP address, it is converted to a standard
format in which each octet is zero-padded in decimal to be three digits long,
and the periods are replaced by semicolons. This mapping is applied to
the asset list and the assets in the intermediate fact base. Finally, the
network model specification is generated.

An example mapping file is provided in Fig.~\ref{mapfile}. With a simple
input and a mapping file, an example of the output produced is given in
Fig.~\ref{gen_example1}; without a mapping file, the output is given in
Fig.~\ref{gen_example2}.

\begin{figure}
\begin{lstlisting}
# Example pysand-ag mapfile

[assets]
george_desktop = 129.244.244.150
ssh_client = 10.0.0.1
ssh_server = 10.0.0.2

[conf]
nslookup = no # Don't try to look up hostnames.
\end{lstlisting}
\caption{Example mapfile}
\label{mapfile}
\end{figure}

\begin{figure}
\begin{lstlisting}
network model =
    assets :
        ssh_client;
        ssh_server;
    facts :
        quality:ssh_client,status=up;
        quality:ssh_server,status=up;
        topology:ssh_client->ssh_server,
                 connected_network_ssh;
.
\end{lstlisting}
\caption{Example generated model using a mapfile}
\label{gen_example1}
\end{figure}

\begin{figure}
\begin{lstlisting}
network model =
    assets :
        010_000_000_001;
        010_000_000_002;
    facts :
        quality:010_000_000_001,status=up;
        quality:010_000_000_002,status=up;
        topology:010_000_000_001->010_000_000_002,
                 connected_network_ssh;
.
\end{lstlisting}
\caption{Example generated model without using a mapfile}
\label{gen_example2}
\end{figure}
\section{Conclusions and Future Work}
This paper presents a tool that utilizes passive network sensors and deep
packet inspection to generate network model specifications appropriate for use
in the style of attack graphs developed at the University of Tulsa. The
generated model is not complete, but it does provide a starting point from
which modeling can proceed. Among its most significant weaknesses are a lack
of inclusion of even basic platform and version information (which SAND
can sometimes provide), and its na\"{i}vet\'{e} in regards to connection
topologies.

Future work along these lines should focus upon extracting as much information
as possible from captured packets in order to identify software versions or
operating systems. This is largely a problem of integration, as tools exist
that are capable of performing this analysis already.

Also, integration with the Common Platform Enumeration (CPE) maintained by
MITRE would allow even more standardization of the lexicon of terms. This
direction could possibly include some minor modifications to the underlying
attack graph model to facilitate the inclusion of CPE style version 
specifications. Finally, no effort is made to intuit whether a 
\texttt{connected\_adjacent} or \texttt{connected\_network} topology should
be used. Further research is needed to determine exactly the significance of
the differences between these two connection topology types, and possibly to 
enable the automated passive detection of which topology type is more 
appropriate.
\bibliographystyle{plain}
\bibliography{sand_ag}
\end{document}